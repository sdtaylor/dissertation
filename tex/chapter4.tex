\chapter{AUTOMATED DATA-INTENSIVE FORECASTING OF PLANT PHENOLOGY THROUGHOUT THE UNITED STATES}

\section{Background}

Numerous phenology models have been developed to characterize the timing of major plant events and understand their drivers \citep{chuine2013}. These models are based on the idea that plant phenology is primarily driven by weather, with seasonal temperatures being the primary driver at temperate latitudes \citep{chuine2017, basler2016}. Because phenology is driven primarily by weather, it is possible to make predictions for the timing of phenology events based on forecasted weather conditions. The deployment of seasonal climate forecasts \citep{weisheimer2014}, those beyond just a few weeks, provides the potential to forecast phenology months in advance. This time horizon is long enough to allow meaningful planning and action in response to these forecasts. With well established models, widely available data, and numerous use cases, plant phenology is well suited to serve as an exemplar for near-term ecological forecasting. \renewcommand*{\thefootnote}{}\footnote{Reprinted with permission from Taylor, S.D. and White, E.P., 2019. Automated data-intensive forecasting of plant phenology throughout the United States. BioRxiv 634568 https://doi.org/10.1101/634568}

For decision making purposes, plant phenology forecasts would be most informative when produced at species- and phenophase-levels, over large spatial extents, and at fine spatial resolutions. Currently the only other plant phenology forecast is at a 1° lat/lon resolution, and uses a temperature based spring index which identifies when early-spring phenological events will occur \citep{schwartz2013, carrillo2018}. Forecasting at a higher level of detail is challenging due to the advanced computational tools needed for building and maintaining automatic forecasting systems \citep{welch2019, white2018}. Automated forecasts requires building systems that acquire data, make model-based predictions for the future, and disseminate the forecasts to end-users, all in an automated pipeline \citep{dietze2018, welch2019, white2018}. This is challenging even for relatively small-scale single site projects with one to several species or response variables due to the need for advanced computational tools to support robust automation \citep{welch2019, white2018}. Building an automated system to forecast phenology for numerous species at continental scales is even more challenging due to the large-scale data intensive nature of the analyses. Specifically, because phenology is sensitive to local climate conditions, phenology modeling and prediction should be done at high resolutions \citep{cook2010}. This requires repeatedly conducting computationally intensive downscaling of seasonal climate forecasts and making large numbers of predictions. To make 4 km resolution spatially explicit forecasts for the 78 species  in our study at continental scales requires over 90 million predictions for each updated forecast. To make the forecasts actionable these computational intensive steps need to be repeated in near real-time and disseminated in a way that allows end-users to understand the forecasts and their uncertainties \citep{dietze2018}.

Here we describe an automated near-term phenology forecast system we developed to make continental scale forecasts for 78 different plant species. Starting December 1st, and updated every 4 days, this system uses the latest climate information to make forecasts for multiple phenophases and presents the resulting forecasts and their uncertainty on a dynamic website, \href{https://phenology.naturecast.org/}{https://phenology.naturecast.org/}. Since the majority of plants complete budburst and/or flowering by the summer solstice in mid-June, this results in lead times off up to six months. We describe the key steps in the system construction, including: 1) fitting phenology models, 2) acquiring and downscaling climate data; 3) making predictions for phenological events; 4) disseminating those predictions; and 5) automating steps 2-4 to update forecasts at a sub-weekly frequency. We follow the framework of \cite{welch2019} for describing operationalized dynamic management tools (ie. self-contained tools running automatically and regularly) and describe the major design decisions and lessons learned from implementing this system that will guide improvements to automated ecological forecasting systems. Due to the data-intensive nature of forecasting phenology at fine resolutions over large scales this system serves as a model for large-scale forecasting systems in ecology more broadly. 

\section{Forecasting Pipeline}

\cite{welch2019} break down the process of developing tools for automated prediction into four stages: 1) Acquisition, obtaining and processing the regularly updated data needed for prediction; 2) Prediction, combining the data with models to estimate the outcome of interest; 3) Dissemination, the public presentation of the predictions; and 4) Automation, the tools and approaches used to automatically update the predictions using the newest data on a regular basis. We start by describing our approach to modeling phenology and then describe our approach to each of these stages.

\subsection{Phenology Modeling}

Making large spatial scale phenology forecasts for a specific species requires species level observation data from as much of its respective range as possible  \citep{taylor2018b}. We used data from the USA National Phenology Network (USA-NPN), which collects volunteer based data on phenological events and has amassed over 10 million observations representing over 1000 species. The USA-NPN protocol uses status-based monitoring, where observers answer yes, no, or unsure when asked if an individual plant has a specific phenophase present \citep{denny2014}. Phenophases refer to specific phases in the annual cycle of a plant, such as the presence of emerging leaves, flowers, fruit, or senescing leaves. We used the Individual Phenometrics data product, which provides pre-processed onset dates of individually monitored plants, for the phenophases budburst, flowering, and fall colors for all species with data between 2009 and 2017 \citep{npndata2017}. We only kept yes observations where the individual plant also had a no observation within the prior 30 days and dropped any records where a single plant had conflicting records for phenotype status or more than one series of yes observations for a phenophase in a 12 month period. We built models for species and phenophase combinations with at least 30 observations (Figure \ref{fig-4-1}-B) using daily mean temperature data at the location and time of each observation from the PRISM 4km dataset \citep{prismdata}. We also included contributed models of budburst, flowering, and/or fruiting for 5 species which were not well represented in the USA-NPN dataset (Table \ref{table-c-1}) \citep{prevey2019a, biederman2018}.

For each species and phenophase we fit an ensemble of four models using daily mean temperature as the sole driver (Figure \ref{fig-4-1}-C). The general model form assumes a phenological event will occur once sufficient thermal forcing units accumulate from a specified start day \citep{chuine2013, chuine2017}. The specification of forcing units are model specific, but all are derived from the 24-hour daily mean temperature. In a basic model a forcing unit is the maximum of either 0 or the mean temperature above 0°C (ie. growing degree days). The amount of forcing units required, and the date from which they start accumulating are parameterized for each species and phenophase (see Supplement for details). Ensembles of multiple models generally improve prediction over any single model by reducing bias and variance, and in a phenology context allow more accurate predictions to be made without knowing the specific physiological processes for each species \citep{basler2016, yun2017, dormann2018}. We used a weighted ensemble of four phenology models. We derived the weights for each model within the ensemble using stacking to minimize the root mean squared error on held out test data (100 fold cross-validation) as described in \cite{dormann2018} (see Supplement for details). After determining the weights we fit the core models a final time on the full dataset. Since individual process based phenology models are not probabilistic they do not allow the estimation of uncertainty in the forecasts. Therefore, we used the variance across the five climate models to represent uncertainty (see Prediction). Finally, we also fit a spatially corrected Long Term Average model for use in calculating anomalies (see Dissemination). This uses the past observations in a linear model with latitude as the sole predictor (see Supplement). 

In our pipeline 190 unique phenological models (one for each species and phenophase combination, see Table \ref{table-c-1}) needed to be individually parameterized, evaluated, and stored for future use. To consolidate all these requirements we built a dedicated software package written in Python, pyPhenology, to build, save, and load models, and also apply them to gridded climate datasets \citep{taylor2018a}. The package also integrates the phenological model ensemble so that the four sub-models can be treated seamlessly as one in the pipeline. After parameterizing each model, its specifications are saved in a text based JSON file that is stored in a git repository along with a metadata file describing all models (Figure \ref{fig-4-1}-D). This approach allows for the tracking and usage of hundreds of models, allowing models to be easily synchronized across systems, and tracking versions of models as they are updated (or even deleted).  

\subsection{Acquisition and Downscaling of Climate Data}

Since our phenology models are based on accumulated temperature forcing, making forecasts requires information on both observed temperatures (from Nov. 30 of the prior year up to the date a forecast is made) and forecast temperatures (from the forecast date onward). For observed data we used 4km 24-hour daily mean temperature from PRISM, a gridded climate dataset for the continental U.S.A. which interpolates on the ground measurements and is updated daily \citep{prismdata}. These observed data are saved in a netCDF file, which is appended with the most recent data every time the automated forecast is run. For climate forecasts we used the Climate Forecast System Version 2 (CFSv2;  a coupled atmosphere-ocean-land global circulation model) 2-m temperature data, which has a 6-hour timestep and a spatial resolution of 0.25 degrees latitude/longitude \citep{saha2014}. CFSv2 forecasts are projected out 9 months from the issue date and are updated every 6 hours. The five most recent climate forecasts are downloaded for each updated phenology forecast to accommodate uncertainty (see Prediction).

Because the gridded climate forecasts are issued at large spatial resolutions (0.25 degrees), this data requires downscaling to be used at ecologically relevant scales \citep{cook2010}. A downscaling model relates observed values at the smaller scale to the larger scale values generated by the climate forecast during a past time period. We regressed these past conditions from a climate reanalysis of CFSv2 from 1995-2015 \citep{saha2010} against the 4km daily mean temperature from the PRISM dataset for the same time period \citep{prismdata} to build a downscaling model using asynchronous regression (Figure \ref{fig-4-1}, E-G). The CFSv2 data is first interpolated from the original 0.25 degree grid to a 4km grid using distance weighted sampling, then an asynchronous regression model is applied to each 4km pixel and calendar month (see Appendix C for details) \citep{stoner2013}. The two parameters from the regression model for each 4 km cell are saved in a netCFD file by location and calendar month (Figure \ref{fig-4-1}-H). This downscaling model, at the scale of the continental U.S.A., is used to downscale the most recent CFSv2 forecasts to a 4km resolution during the automated steps. 

We used specialized Python packages to overcome the computational challenges inherent in the large CFSv2 climate dataset \citep{python}. The climate forecast data for each phenology forecast update is 10-40 gigabytes, depending on the time of year (time series are longer later in the year). While it is possible to obtain hardware capable of loading this dataset into memory, a more efficient approach is to perform the downscaling and phenology model operations iteratively by subsetting the climate dataset spatially and performing operations on one chunk at a time. We used the python package xarray \citep{xarray}, which allows these operations to be efficiently performed in parallel through tight integration with the dask package \citep{dask}. The combination of dask and xarray allows the analysis to be run on individual workstations, stand alone servers, and high performance computing systems, and to easily scale to more predictors and higher resolution data.

\subsection{Prediction}

The five most recent downscaled forecasts are each combined with climate observations to make a five member ensemble of daily mean temperature across the continental USA. These are used to make predictions using the phenology model for each species and phenophase (Figure \ref{fig-4-1}-M). Each climate ensemble member is a 3d matrix of latitude $\times$ longitude $\times$ time at daily timesteps extending from Nov. 1 of the prior year to 9 months past the issue date. The pyPhenology package uses this object to make predictions for every 4 km grid cell in the contiguous United States, producing a 2d matrix (latitude $\times$ longitude) where each cell represents the predicted Julian day of the phenological event. This results in approximately half a million predictions for each run of each phenology model and 90 million predictions per run of the forecasting pipeline. The output of each model is cropped to the range of the respective species \citep{little1999} and saved as a netCDF file (Figure \ref{fig-4-1}-N) for use in dissemination and later evaluation. 

An important aspect of making actionable forecasts is providing decision makers with information on the uncertainty of those predictions \citep{dietze2018}. One major component of uncertainty that is often ignored in near-term ecological forecasting studies is the uncertainty in the forecasted drivers. We incorporate information on uncertainty in temperature, the only driver in our phenology models, using the CFSv2 climate ensemble (Figure \ref{fig-4-1}-I; see Acquisition). The members of the climate ensemble each produce a different temperature forecast due to differences in initial conditions \citep{weisheimer2014}. For each of the five climate members we make a prediction using the phenology ensemble, and the uncertainty is estimated as the variance of these predictions (see Supplement). This allows us to present the uncertainty associated with climate, along with a point estimate of the forecast, resulting in a range of dates over which a phenological event is likely to occur.

\subsection{Dissemination}

To disseminate the forecasts we built a website that displays maps of the predictions for each unique species and phenophase (\href{https://phenology.naturecast.org/}{https://phenology.naturecast.org/}; Figure \ref{fig-4-1}-Q; Figure \ref{fig-4-2}). We used the Django web framework and custom JavaScript to allow the user to select forecasts by species, phenophase, and issue date (Figure \ref{fig-4-2}-D). The main map shows the best estimate for when the phenological event will occur for the selected species (Figure \ref{fig-4-2}-A). Actionable forecasts also require an understanding of how much uncertainty is present in the prediction \citep{dietze2018}, because knowing the expected date of an annual event such as flowering isn’t particularly useful if the confidence interval stretches over several months. Therefore we also display a map of uncertainty quantified as the 95\% prediction interval, the range of days within which the phenology event is expected to fall 95\% of the time (Figure \ref{fig-4-2}-C). Finally, to provide context to the current years predictions, we also map the predicted anomaly (Figure \ref{fig-4-2}-B). The anomaly is the difference between the predicted date and the long term, spatially corrected average date of the phenological event (Figure \ref{fig-4-1}-O). 

\subsection{Automation}

All of the steps in this pipeline, other than phenology and downscaling model fitting, are automatically run every 4 days. To do this we use a cron job running on a local server. Cron jobs automatically rerun code on set intervals. The cron job initiates a python script which runs the major steps in the pipeline. First the latest CFSv2 climate forecasts are acquired, downscaled, and combined with the latest PRISM climate observations (Figure \ref{fig-4-1}, I-L). This data is then combined with the phenology models using the pyPhenology package to make predictions for the timing of phenological events (Figure \ref{fig-4-1}, M-N). These forecasts are then converted into maps and uploaded to the website (Figure \ref{fig-4-1}, O-Q). To ensure that forecasts continue to run even when unexpected events occur it is necessary to develop pipelines that are robust to unexpected errors and missing data, and are also informative when failures inevitably do happen \citep{welch2019}. We used status checks and logging to identify and fix problems and separated the website infrastructure from the rest of the pipeline. Data are checked during acquisition to determine if there are data problems and when possible alternate data is used to replace data with issues. For example, members of the CFSv2 ensemble sometimes have insufficient time series lengths. When this is the case that forecast is discarded and a preceding climate forecast obtained. With this setup occasional errors in upstream data can be ignored, and larger problems identified and corrected with minimal downtime. To prevent larger problems from preventing access to the most recent successful forecasts the website is only updated if all other steps run successfully. This ensures that user of the website can always access the latest forecasts.

Software packages used throughout the system include, for the R language, ggplot2 \citep{ggplot2}, raster \citep{rasterR}, prism \citep{prismR}, sp \citep{sp1}, tidyr \citep{tidyr}, lubridate \citep{lubridate}, and ncdf4 \citep{ncdf4}. From the python language we also utilized xarray \citep{xarray}, dask, \citep{dask}, scipy \citep{scipy}, numpy \citep{numpy}, pandas \citep{pandas}, and mpi4py \citep{mpi4py}. The code as well as 2019 forecasts and observations (see Evaluation) are also permanently archived on Zenodo (\href{https://doi.org/10.5281/zenodo.2577452}{https://doi.org/10.5281/zenodo.2577452}).

\section{Evaluation}

A primary advantage of near-term forecasts is the ability to rapidly evaluate forecast proficiency, thereby shortening the model development cycle \citep{dietze2018}. Phenological events happen throughout the growing season, providing a consistent stream of new observations to assess. We evaluated our forecasts (made from Dec. 1, 2018 thru May 1, 2019) using observations from the USA-NPN from Jan. 1, 2019 through May 8, 2019 and subset to species and phenophases represented in our system \citep{npndata2019}. This resulted in 1581 phenological events that our system had forecast for (588 flowering events, 991 budburst events, and 2 fall coloring across 65 species, see Supplement). For each forecast issue date we calculated the root mean square error (RMSE) and average forecast uncertainty for all events and all prior issue dates. We also assessed the distribution of absolute errors ($\widehat{DOY} - DOY$) for a subset of issue dates (approximately two a month).

Forecast RMSE and uncertainty both decreased for forecasts with shorter lead time (i.e. closer to the date the phenological event occurred), also known as the forecast horizon (Figure \ref{fig-4-4}) \citep{petchey2015}. Forecasts issued at the start of the year (on Jan. 5, 2019) had a RMSE of 20.9 days, while the most recent forecasts (on May 5, 2019) had an RMSE of only 18.8 days. The average uncertainty for the forecasts were 7.6 and 0.2 days respectively for Jan. 5, and May 5. Errors were normally distributed with a small over-prediction bias (MAE values of 6.8 - 12.1, Figure \ref{fig-4-5}). This bias also decreased as spring progressed. These results indicate a generally well performing model, but also one with significant room for improvement that will be facilitated by the iterative nature of the forecasting system.

\section{Discussion}

We created an automated plant phenology forecasting system that makes forecasts for 78 species and 4 different phenophases across the entire contiguous United States. Forecasts are updated every four days with the most recent climate observations and forecasts, converted to static maps, and uploaded to a website for dissemination. We used only open source software and data formats, and free publicly available data. While a more comprehensive evaluation of forecast performance is outside the scope of this paper, we note that the majority of forecasts provide realistic phenology estimates across known latitudinal and elevational gradients (Figure \ref{fig-4-2}), and forecast uncertainty and error decreases as spring progresses (Figure \ref{fig-4-4}). While there is a bias from over-estimating phenological events, estimates were on-average within 2-3 weeks of the true dates throughout the spring season.  

Developing automated forecasting systems in ecology is important both for providing decision makers with near real-time predictions and for improving our understanding of biological systems by allowing repeated tests of, and improvements to, ecological models \citep{dietze2018, welch2019, white2018}. Facilitating the development of these kinds of systems requires both active development of a variety of forecasting systems and discussion of the tools, philosophies, and challenges involved in their development. These explorations and conversations will advance our understanding of how to most effectively build ecological forecasting systems and lower the entry barrier of operationalizing ecological models for decision making. This will also support our understanding of the general problem of automated ecological forecasts to allow the development of standardized methods, data formats, and software packages to help support the implementation of new ecological forecasts and facilitate synthetic analyses across orecasts made for an array of outcomes across many ecosystems. 

Automated forecasting systems typically involve multiple major steps in a combined pipeline. We found that breaking the pipeline into modular chunks made maintaining this large number of components more manageable \citep{welch2019, white2018}. For generalizable pieces of the pipeline we found that turning them into software packages eased maintenance by decoupling dependencies and allowing independent testing. Packaging large components also makes it easier for others to use code developed for a forecasting system. The phenology modelling packge, pyPhenology, was developed for the current system, but is generalized for use in any phenological modelling study \citep{taylor2018a}. We also found it useful to use different languages for different pieces of the pipeline. Our pipeline involved tasks ranging from automatically processing gigabytes of climate data to visualizing results to disseminating those results through a dynamic website. In such a pipeline no single language will fit all requirements, thus we made use of the strengths of two languages (Python and R) and their associate package ecosystems. Interoperability is facilitated by common data formats (csv and netCDF files), allowing scripts written in one language to communicate results to the next step in the pipeline written in another language. 

This phenology forecasting system currently involves 190 different ensemble models, one for each species and phenological stage, each composed of 4 different phenology sub-models and their associated weights for a total of 760 different models. This necessitates having a system for storing and documenting models, and subsequently updating them with new data and/or methods over time. We stored the fitted models in JSON files (a open-standard text format). We used the version control system git to track changes to these text based model specifications. While git was originally designed tracking changes to code, it can also be leveraged for tracking data of many forms, including our model specifications \citep{ram2013, bryan2018, yenni2019}. Managing many different models, including different versions of those models and their associate provenance, will likely be a common challenge for ecological forecasting \citep{white2018} as one of the goals is iteratively improving the models.

The initial development of this system has highlighted several potential areas for improvement. First, the data-intensive nature of this forecasting system provides challenges and opportunities for disseminating results. Currently static maps show the forecast dates of phenological events across each species respective range. However this only answers one set of questions and makes it difficult for others to build on the forecasts. Additional user interface design, including interactive maps and the potential to view forecasts for a single location, would make it easier to ask other types of questions such as “Which species will be in bloom on this date in a particular location?”. User interface design is vital for successful dissemination, and tools such the python package Django used here, or the R packages Shiny and Rmarkdown provide flexible frameworks for implementation \citep{white2018, welch2019}. In addition it would be useful to provide access to the raw data underlying each forecast. The sheer number of forecasts makes the bi-weekly forecast data relatively large, presenting some challenges for dissemination through traditional ecological archiving services like Dryad (\href{https://datadryad.org}{https://datadryad.org}) and Zenodo (\href{https://zenodo.org}{https://zenodo.org}). If stored as csv files every forecast would have generated 15 GB of data. We addressed this by storing the forecasts in compressed netCDF files, which are optimized for large-scale mutli-dimensional data and in our case are 300 times smaller than the csv files (50 MB/forecast).

In addition to areas for improvement in the forecasting system itself, its development has highlighted areas for potential improvement in phenology modeling. Other well-known phenological drivers could be incorporated into the models, such as precipitation and daylength. Precipitation forecasts are available from the CFSv2 dataset, though their accuracy is considerably less than temperature forecasts \citep{saha2014}. Other large-scale phenological datasets, such as remotely-sensed spring greenup could be used to constrain the species level forecasts made here \citep{melaas2016}. Our system does not currently integrate observations about how phenology is progressing within a year to update the models. USA-NPN data are available in near real-time after they are submitted by volunteers, thus there is opportunity for data assimilation of phenology observations. Making new forecasts with the latest information not only on the current state of the climate, but also on the current state of the plants themselves would likely be very informative \citep{luo2011, dietze2017}. For example, if a species is leafing out sooner than expected in one area it is likely that it will also leaf out sooner than expected in nearby regions. This type of data assimilation is important for making accurate forecasts in other disciplines including meteorology \citep{bauer2015, carrassi2018}. However, process based plant phenology models were not designed with data assimilation in mind \citep{chuine2013}. \cite{clark2014b} built a bayesian hierarchical phenology model of budburst which incorporates the discrete observations of phenology data. This could serve as a starting point for a phenology forecasting model that incorporates data assimilation and allows species with relatively few observations to borrow strength from species with a large number of observations. The model from \cite{clark2014b} also incorporates all stages of the bud development process into a continuous latent state, thus there is also potential for forecasting the current phenological state of plants, instead of just the transition dates as is currently done in this forecast system. 

%%%%%%%%%%%%%%%%%%%%%%%
%%%%%%%%%%%%%%%%%%%%%%%
%% Begin figures
%%%%%%%%%%%%%%%%%%%%%%%
%%%%%%%%%%%%%%%%%%%%%%%
\newpage

%%%%%%%%%%%%%%%%%%%%%%
%% Figure 1 pipeline schematic
%%%%%%%%%%%%%%%%%%%%%%

\begin{figure}
	\centering
		%\includegraphics[scale=0.4]{images/figure_4-1_schematic.png}
		\includegraphics[scale=0.3]{images/figure_filler.jpg}
	\caption[Flowchart of initial model building and automated pipeline steps]{Flowchart of initial model building and automated pipeline steps. Letters indicate the associate steps discussed in the main text.} \label{fig-4-1}
\end{figure}

%%%%%%%%%%%%%%%%%%%%%%
%% Figure 2 website screenshot
%%%%%%%%%%%%%%%%%%%%%%

\begin{figure}
	\centering
		%\includegraphics[scale=0.4]{images/figure_4-2_website_screenshot.png}
		\includegraphics[scale=0.3]{images/figure_filler.jpg}
	\caption[Screenshot of the forecast presentation website]{Screenshot of the forecast presentation website. Shown is the forecast for the leaf out of \textit{Acer saccharinum} in Spring, 2019, issued on Feburary 21, 2019. The maps represent the predicted date of leaf out (A), the anomaly compared to prior years (B), and the 95\% confidence interval (C). In the upper right is the interface for selecting different species, phenophases, or forecast issue dates via drop down menus (D).} \label{fig-4-2}
\end{figure}

%%%%%%%%%%%%%%%%%%%%%%
%% Figure 3 population peak errors
%%%%%%%%%%%%%%%%%%%%%%
\begin{figure}
	\centering
		%\includegraphics[scale=0.6]{images/figure_4-3_map_figure.png}
		\includegraphics[scale=0.3]{images/figure_filler.jpg}
	\caption[Locations of phenological events which have occurred in 2019]{Locations of phenological events which have occurred in 2019. Events between Jan. 1, 2019 and May 5, 2019 obtained from the USA National Phenology Network (blue circles), and all sampling locations in the same dataset (red points). Four individual plants are highlighted, where the solid line indicates the predicted event date as well as 95\% confidence interval for a specified forecast issue date, and the dashed line indicates the observed event date. The x-axis corresponds to the date a forecast was issued, while the y-axis is the date flowering or budburst was predicted to occur. For example: on Jan. 1, 2019 the \textit{P. tremuloides} plant was forecasted to flower sometime between March, 29 and April, 24 (solid lines), while the actual flowering date was March 18 (dashed line).} \label{fig-4-3}
\end{figure}

%%%%%%%%%%%%%%%%%%%%%%
%% Figure 4 Metric timeseries (RMSE & SD)
%%%%%%%%%%%%%%%%%%%%%%
\begin{figure}
	\centering
		%\includegraphics[scale=0.4]{images/figure_4-4_metric_timeseries.png}
		\includegraphics[scale=0.3]{images/figure_filler.jpg}
	\caption[The root mean square error and the average uncertainty of forecasts]{The root mean square error and the average uncertainty of forecasts. Forecasts were issued between Dec. 2, 2018 and May 5, 2019 for 1581 phenological events representing 65 species. } \label{fig-4-4}
\end{figure}

%%%%%%%%%%%%%%%%%%%%%%
%% Figure 5 Error timeseries (MAE Boxplots)
%%%%%%%%%%%%%%%%%%%%%%
\begin{figure}
	\centering
		%\includegraphics[scale=0.3]{images/figure_4-5_error_timeseries.png}
		\includegraphics[scale=0.3]{images/figure_filler.jpg}
	\caption[Distribution of absolute errors]{Distribution of absolute errors. The errors (prediction - observed) are for 1581 phenological events for 11 selected issue dates. Labels indicate the mean absolute error (MAE).} \label{fig-4-5}
\end{figure}